
%------------------------------------------------------------------------------------------
\question [] % Limites

Para cada uma das funções abaixo, determine o domínio e calcule os limites pedidos. Lembre-se de justificar cada passo, usando as Propriedades dos Limites.

\begin{multicols}{2}
  \begin{itemize}
    \item[a)] $f(x) = \dfrac{x^2 - 6x + 9}{x-3}$ ; \ \ $\lim\limits_{x\rightarrow 3}f(x)$
      \vspace{0.3cm}
    \item[b)] $f(x) = \dfrac{2x+8}{x^2 + x - 12}$ ; \ \ $\lim\limits_{x\rightarrow -4}f(x)$
      \vspace{0.3cm}
    \item[c)] $f(x) = \dfrac{x^2 - 3x - 10}{x^2 - 10x + 25}$ ; \ \ $\lim\limits_{x\rightarrow 5}f(x)$
      \vspace{0.3cm}
  \end{itemize}
\end{multicols}

%------------------------------------------------------------------------------------------
\question [] % Limites
O que há de errado com a equação a seguir?
$$\dfrac{x^2 + x - 6}{x-2} = x+3$$
Em vista do fato anterior, explique por que a equação
$$\lim\limits_{x\rightarrow 2}\dfrac{x^2 + x - 6}{x-2} = \lim\limits_{x\rightarrow 2}x+3$$
está correta.

%------------------------------------------------------------------------------------------
