
% Habilita a língua portuguesa
\usepackage[utf8]{inputenc}  % Codificação de entrada para UTF-8
\usepackage[T1]{fontenc}        % Suporte para caracteres acentuados
\usepackage[brazilian]{babel}   % Adapta o documento para o português do Brasil

% Define novos comandos
\DeclareMathOperator{\sen}{sen}
\DeclareMathOperator{\arcsen}{arcsen}
\DeclareMathOperator{\senh}{senh}
\DeclareMathOperator{\arsenh}{arsenh}


% Redefine os comandos para português
\renewcommand{\sin}{\operatorname{sen}\,}
\renewcommand{\sinh}{\operatorname{senh}\,}
\renewcommand{\arcsin}{\operatorname{arcsen}\,}

% Renomeando teoremas para portugues
\theoremstyle{plain}
\newtheorem{teorema}{Teorema}[section]
\newtheorem{axioma}[teorema]{Axioma}
\newtheorem{proposicao}[teorema]{Proposição}
\newtheorem{lema}[teorema]{Lema}
\newtheorem{corolario}[teorema]{Corolário}

% Renomeando definiçoes e exemplos
\theoremstyle{definition}
\newtheorem{definicao}[teorema]{Definição}
\newtheorem{exemplo}[teorema]{Exemplo}
\newtheorem{exercicio}[teorema]{Exerc\'{\i}cio}
% \newtheorem{question}{Questão}

% Renomeando Observações
\theoremstyle{remark}
\newtheorem{observacao}[teorema]{Observação}
\newtheorem{observacoes}[teorema]{Observações}
\newtheorem{notacao}[teorema]{Notação}

% Renomeando Provas e Soluções
\newenvironment{prova}{\vspace{.1cm}\noindent {\red{\bf Prova.}}}{
  $\cqd$\vspace{2mm}}

\newenvironment{solucao}{\vspace{.1cm}\noindent {\green{\bf Solução.}}\footnotesize}{
  $\cqd$\vspace{2mm}}

% Usa o sistema internacional de memidas
\usepackage{siunitx}   % Formata unidades e números no Sistema Internacional de Unidades (SI)

% coloca o ponto como separador de milhar
\sisetup{
  group-four-digits = true,
  group-separator = {.}
}

%transforma o simbolo decimal . em , no modo matemático.
% \DeclareMathSymbol{.}{\mathord}{letters}{"3B}

