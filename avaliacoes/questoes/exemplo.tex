Dado o triângulo cujos vértices são $A(1,1)$, $B(4,0)$ e $C(3,4)$.
\begin{parts}
  \part Encontre a projeção do lado $\overline{AC}$ sobre o lado $\overline{AB}$.
  \part Ache o pé da altura relativa ao vértice $C$.
  \part Calcule a área do triângulo.
\end{parts}

\begin{solution}
  Sejam os vértices do triângulo \(A(1,1)\), \(B(4,0)\) e \(C(3,4)\).

  \begin{parts}
    \part \textbf{Projeção do lado \(\overline{AC}\) sobre \(\overline{AB}\)}:

    A projeção de \(\overrightarrow{AC} = (2,3)\) sobre
    \(\overrightarrow{AB} =(3,-1)\) é dada por:
    \[
      \text{proj}_{\overrightarrow{AB}} \overrightarrow{AC} =
      \frac{\overrightarrow{AC} \cdot \overrightarrow{AB}}{\overrightarrow{AB}
        \cdot \overrightarrow{AB}} \overrightarrow{AB} = \left( \frac{9}{10},
      -\frac{3}{10} \right)
    \]

    \part \textbf{Pé da altura relativa ao vértice \(C\)}:
    O pé da altura relativa ao vértice \(C\) é o ponto
    \[
      P= A + \text{proj}_{\overrightarrow{AB}} = (1,1) + \left( \frac{9}{10},
      -\frac{3}{10} \right)=\left( \frac{19}{10}, \frac{7}{10} \right)
    \]

    \part \textbf{Área do triângulo}:
    \[
      \text{Área} = \frac{1}{2} \| {\overrightarrow{AB}}\|\cdot\|
      {\overrightarrow{CP}}\| = \frac{1}{2}\cdot \sqrt{10}\cdot
      \frac{11}{\sqrt{10}}=\frac{11}{2}=5,5.
    \]
  \end{parts}
\end{solution}
